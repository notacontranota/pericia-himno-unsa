\chapter{Prólogo}
\label{cap:prologo}

La propuesta original\footnote{Expediente ingresado el 24 de febrero de 2023 a horas 16:00 bajo el título \emph{Propuesta para una pericia de una melodía atribuida a Gustavo «Cuchi» Leguizamón}.} elaborada por mí para la \textsc{Universidad Nacional de Salta} para hacer una pericia sobre una línea melódica presente en la memoria de uno de los hijos del compositor salteño \textsc{Gustavo «Cuchi» Leguizamón} se vio necesariamente modificada cuando, por acción de la licenciada \textsc{Lucrecia Coscio}\footnote{\textsc{Lucrecia Coscio} es coordinadora del \textsc{Centro Cultural Holver Martínez Borelli}, dependiente de la Secretaría de Extensión Universitaria de la UNSa. }, se encontró una caja con documentación relacionada con el concurso para la letra de un himno a la Institución que la Universidad había llevado a cabo hace más de cuarenta años. En tal documentación, entre otras, se halló una partitura que, en parte, respondía a lo que \textsc{Juan Martín Leguizamón}, hijo del compositor, nos había ya referido como la melodía del \emph{Himno a la Universidad Nacional de Salta} que su padre había compuesto. Lo que en principio iba a ser un análisis de las estructuras intermedias y de superficie\footnote{\emph{Estructuras intermedias y de superficie} son términos propios de las técnicas de análisis schenkeriano.} de dicha melodía y de las posibles formas de armonizarla, pasó, obligadamente, a ser el análisis de una partitura en la que las armonías están completamente definidas y la línea melódica coincide solamente en forma parcial a lo aportado por el hijo del «Cuchi». Esas diferencias ---sin dudas significativas--- representan una oportunidad para poder mostrar un aspecto poco conocido de la obra de Leguizamón: su obra musical no son solamente zambas y chacareras en compás de \hbox{\compas{6}{8}(\compas{3}{4}).}\footnote{El compás de \hbox{\compas{6}{8}(\compas{3}{4}).} es el patrón rítmico complejo de mucha música latinoamericana que combina tanto en la sucesión como en la simultaneidad patrones de dos pulsos (\compas{6}{8}) y tres pulsos (\compas{3}{4}).} A la vez también es una oportunidad para reflexionar mínimamente sobre lo que es ---y lo que no es--- la memoria.

La partitura encontrada con la documentación referente al concurso de 1982 organizado por la Universidad \emph{para elegir una letra} para el futuro Himno ---concurso que al igual que el celebrado un año antes fuera declarado desierto--- contiene la letra del concursante que se presentara con el seudónimo \textsc{Budotizo}. Al no ser esta letra ganadora del concurso, con criterio que compartimos plenamente, las actuales autoridades de la Institución piensan en una nueva letra. Para tal fin se nos hace necesario analizar tanto las acentuaciones y la forma musical en relación a las acentuaciones y la forma de esta poesía como así también advertir de las divergencias entre los acentos musicales y poéticos que en una letra bien confeccionada para esta música no deben estar.

El \textsc{«Cuchi» Leguizamón}, en conocida entrevista televisiva y  con lenguaje coloquial, nos revela parte de su pensamiento musical, consistente en incluir en la simultaneidad dos enfoques de un mismo objeto, algo que en técnica de composición se conoce como \emph{bitonalidad}. Hemos considerado que en el marco del presente informe cabía desarrollar un capítulo de carácter más técnico, que con seguridad requiere de altas competencias en teoría musical para abordar con total profundidad su lectura, y no sólo cabía sino que hasta necesario se hace, aunque más no sea en ciertos límites, mostrar en términos teóricos lo que de las obras musicales de Leguizamón se desprende como pensamiento y recurso expresivo.

La valoración del hallazgo y rescate de una pieza de uno de los más emblemáticos compositores de la cultura de Salta y la presentación de  posibilidades futuras sobre ella cierran este informe que entre otras cosas nos ha permitido \emph{empezar} a poner en palabras un tanto más rigurosas el modo de hacer de una mente brillante y un espíritu inspirado: la mente y el espíritu de \textsc{Gustavo «Cuchi» Leguizamón}.
