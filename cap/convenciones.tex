\chapter{Algunas convenciones de escritura
musical}
\label{pre:convenciones}

\section*{Grados}

Para los grados melódicos hemos adoptado le convención schenkeriana: números arábigos para los grados melódicos y números romanos para los grados armónicos.

\begin{table}[H]
\centering
\begin{tabular}{@{}lll@{}}
\toprule
\textbf{Grado} & \textbf{Melódico}  & \textbf{Armónico} \\
\midrule
Primer grado   & \grado{1}          & I                 \\
Segundo grado  & \grado{2}          & II                \\
Tercer grado   & \grado{3}          & III               \\
Cuarto grado   & \grado{4}          & IV                \\
Quinto grado   & \grado{5}          & V                 \\
Sexto grado    & \grado{6}          & VI                \\
Séptimo grado  & \grado{7}          & VII               \\
Octavo grado   & \grado{8}          & ---              \\
\bottomrule
\end{tabular}
\caption*{Grados.}\label{pretab:grados}
\end{table}

El bajo en los grados armónicos se escribe como subíndice bajo el grado (en romano) o el nombre de la fundamental. Las notas añadidas a la triada se escriben como superíndice. Ejemplo: la dominante séptima del segundo grado con su tercera alterada en el bajo se cifra \acorde.V.\sostenidotxt3.7.../II, y, en la tonalidad de Fa mayor, también se puede escribir como \acorde.Re.Fa\sostenidotxt.7....

\section*{Ligaduras}

La ligadura semipunteada significa dependencia de la primera respecto a la segunda nota. Ejemplo:

\begin{figure}[H]
\centering
\begin{lilypond}[notime]
\relative {
  \hide Stem
  \slurHalfDashed
  e''4 ( c2 ) \bar "||"
  e,4 ( g2 ) \bar "||"
}
\end{lilypond}
\caption*{Dependencia de una nota.}\label{prefig:dependencia}
\end{figure}

Las ligaduras punteadas se utilizan para marcar \emph{puentes de segundas}. Ejemplo:

\begin{figure}[H]
\centering
\begin{lilypond}
\relative {
  \time 2/4
  \tempo "Tempo di ritorno"
  \partial 8
  f''16 g
  \slurDashed
  a4. ( g16 f g8 ) ( e4 e8
  f4. ) ( e16 d e8 )
}
\end{lilypond}
\caption*{Puente de segundas.}
\label{prefig:puente}
\end{figure}
