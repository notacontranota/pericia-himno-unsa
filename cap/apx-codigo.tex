\chapter{Código fuente de la partitura}
\label{cap:codigo}

Para la edición de la partitura hemos utilizado \texttt{GNU/LilyPond}\footnote{\texttt{\textbf{GNU/LilyPond}} es software libre, descargable desde \url{www.lilypond.org}. Está disponible para los sistemas operativos más importantes. \texttt{\textbf{Hacklily}}, \url{https://www.hacklily.org/} es una implementación web de LilyPond que permite utilizar este editor de partituras en cualquier dispositivo con un navegador web y una conexión a internet. Copiando y pegando el código acá facilitado se obtiene un archivo PDF de la partitura del Himno a la UNSa de Leguizamón y Pérez.}. El código para compilarla, a continuación.

{\scriptsize
\begin{verbatim}
\version "2.24.1"

global = {
  \key f \major
  \time 2/4
  \tempo "Introducción" 4 = 60
  \partial 4 s4
  s2*16
  s4.
  \tempo "Canto" s8
  \bar "||"
  s2*23
  \bar "|."
}

melodia = \relative {
  r4
  R2*16
  r4. c'8
  f4. f8
  f d e f
  g4 g8 f
  e4. c8
  a'4. a8
  a f g a
  bes4 bes8 a
  g4. c8
  f4. f8
  f d e f
  e4 c8 a
  c4. d8
  bes4 bes
  bes4. c8
  a g a f
  g d e c
  f4. f8
  f d e f
  bes4. bes8
  bes g a bes
  c4 f8 g
  d4 e
  f2
}

manoderecha = \relative {
  \tuplet 3/2 {c'''8 a g}
  \acciaccatura gis8 a d,4.
  \acciaccatura gis8 a des,4.
  \acciaccatura fis8 g c,4.
  e,16 f a c \tuplet 3/2 {g'8 f e}
  g bes,4.
  \tuplet 3/2 {<e, a>8 bes' c} \tuplet 3/2 {<f, bes d> f' e}
  <d, a' c>4 ~ \tuplet 3/2 {<d a' c>8 a' bes}
  <es, fis c'>4 \tuplet 3/2 {c''8 a g}
  \acciaccatura gis8 a d,4.
  \acciaccatura gis8 a des,4.
  \acciaccatura fis8 g c,4.
  e,16 f a c \tuplet 3/2 {g'8 f e}
  g bes,4.
  \tuplet 3/2 {<e, a>8 bes' c} <f, bes d> a'
  <f, c' f>4 ~ \tuplet 3/2 {<f c' f>8 a bes}
  <es, fis c'>4 ~ \tuplet 3/2 {<es fis c'>8 a bes}
  <e, a c>4. c8
  f4. f8
  f d e f
  <b, g'>4 <b g'>8 f'
  e4. c8
  <e a>4. <e a>8
  <e a> <c f> <e g> a
  <f bes>4 <f bes>8 a
  <e g>4. c'8
  <f, d' f>4. q8
  <b f'> d <b e> f'
  <f, e'>4 <g c>8 <e a>
  <es fis a c>4. d'8
  <f, bes>4 q
  bes4. c8
  <e, a>8 g <e a> f
  <b, g'> d e c
  f4. f8
  f8 d e f
  <f bes>4. q8
  q g <e a> bes'
  <e, a c>4 <c' f>8 g'
  <f, b d>4 <e gis e'>
  <f a c f>2
}

manoizquierda = \relative {
  \clef bass
  r4
  <g,,g'>4 <d''' f bes>
  <c, bes' e> <c' bes'>
  <f,, c' a'> <d'' a'>
  <d, a' c> <d c' f>
  <g, d'bes'> d''
  <c, bes' d> <g' d'>
  <f, c' a'> d'
  <d a' c> d'
  <g,,, g'> <d''' f bes>
  <c, bes' e> <c' bes'>
  <f,, c' a'> <d'' a'>
  <d, a' c> <d c' f>
  <g, d'bes'> d''
  <c, bes' d> <g' d'>8 <c, bes' e>
  <f, d' a'>4 d''
  <d, c'> d'
  <c, bes' d>2
  f,8 c' a'4
  <d, a' c>2
  <g, f'>4 g'
  c,8 bes' d4
  f,,8 c' d e
  f a ~ a4
  <g d'>8 cis <g d'> cis
  <c, bes' d>4 c'
  d,,8 a' f' c'
  b, gis' d'4
  <f,, c' a'>2
  <d' c'>
  <g d'>8 cis <g d'> cis
  <c, c'!>4 d'
  <f,, c' a'> <d' a' c>
  <g, f'> <c bes'?>
  f,8 c' a'4
  <d, a' c>2
  <g d'>4 cis
  <c, bes' d> c'
  <f,, c' a'> <d' a' f'>
  <g, d' b'> <c bes' d>
  <f c' d>2
}

versos = \lyricmode {
  Es -- toy so -- bre mi tie -- rra
  pa -- ra be -- ber
  el ai -- re que me~a -- lien -- ta
  y~e -- se li -- cor
  que los an -- ti -- guos dio -- ses
  o -- fren -- da -- rán
  has -- ta que~al fin,
  des -- nu -- da~el al -- ma~en -- su~es -- plen -- dor,
  re -- tor -- ne~ha -- cia la pu -- ra~e -- sen -- cia~en
  las ver -- tien -- tes lím -- pi -- das del sa -- ber.
}

\header {
  title = "Himno a la Universidad Nacional de Salta"
  dedication = \markup{\italic "rescatado del archivo universitario"}
  %instrument = "Canto y Piano"
  composer = \markup{ \caps "Gustavo «Cuchi» Leguizamón"}
  poet = \markup{ \caps "Miguel Ángel Pérez"}
  tagline = ""
}

#(set-global-staff-size 20)

\score {
  <<
    \new Staff << \melodia \global >> \addlyrics \versos
    \new PianoStaff {
      <<
        \new Staff << \manoderecha \global >>
        \new Staff << \manoizquierda \global >>
      >>
    }
  >>
  \layout {
    indent = 0\cm
  }
  \midi {}
}

\markup {
  \lower #10
  \fontsize #3 {
    \hspace #28
    \column {
      \line { Estoy sobre mi tierra}
      \line { \null }
      \line { el aire que me alienta }
      \line { \null }
      \line { que los antiguos dioses }
      \line { \null }
      \line { hasta que al fin, }
      \line { desnuda el alma en su esplendor, }
      \line { retorne hacia la pura esencia }
      \line { en las vertientes límpidas del saber. }
    }
    \hspace #-20
    \column {
      \line { \null }
      \line { para beber }
      \line { \null }
      \line { y ese licor }
      \line { \null }
      \line { ofrendarán }
    }
  }
}
\end{verbatim}
}
