\chapter{Caminos posibles}
\label{cap:caminos-posibles}

Hay una buena noticia: una pieza del «Cuchi» Leguizamón ha sido rescatada. Para que ella pueda ser considerada el Himno de la Universidad de Salta hay que recorrer alguno de posibles caminos para allanar las dificultades que la pieza, por algunas características que posee, impone.

\section{Tonalidad y registro}
\label{sec:tonalidad-registro}

Entre esas características está, en primer lugar, la tesitura de la voz humana para la que fue escrita, que no condice con el registro de la media de la población de Salta (ni la mundial). Adaptar la pieza transportándola un tono y medio abajo, a la tonalidad de Re mayor ---recomendable para orquesta---, o dos tonos abajo, a la tonalidad de Re\bemoltxt\ mayor ---recomendable para banda--- es imprescindible para que la línea melódica no quede accesible solamente a voces educadas musical y técnicamente. Sin embargo, el transporte no es gratis: lo que en Fa mayor es tocable en el piano, tanto en Re como en Re\bemoltxt\ mayor cobra dificultades, especialmente en los intervalos de décima, ya que no es lo mismo tocar el intervalo \musncp[voffset=-1pt]{\key f \major \clef bass <f, a>} que tocar el intervalo \hbox{\musncp[voffset=-2pt]{\key d \major \clef bass <d, fis>},} el cual requiere de una mano más grande. Lo mismo sucede con el transporte a Re\bemoltxt\ mayor. De todos modos, estos transportes, lo recordemos, están siendo sugeridos para versiones orquestadas o arregladas para banda militar u orquesta sinfónica. Claro está que no deja de ser un dilema ---y no es nada recomendable--- tener o no versiones de un himno en diferentes tonalidades. Debe elegirse una, y adaptar las cuestiones técnicas a las diversas y posibles instrumentaciones.

\subsection{Versión transportada a Re mayor}
\label{subsec:transporte-re}

\lilypondfile[staffsize=11]{part/leguizamon-himno_unsa-re.ly}

\subsection[Versión transportada a Re\bemoltxt\ mayor]{Versión transportada a Re\bemol\ mayor}
\label{subsec:transporte-reb}

\lilypondfile[staffsize=11]{part/leguizamon-himno_unsa-reb.ly}


\section{Orquestación, o arreglo}
\label{sec:orquestacion-arreglo}




% TODO %%%%%%%%%%%%%%%%%%%%%%%%%%%%%%%%%%%%%%%%%%%%%
% Memoria como acto del presente que involucra la
% imaginación y las competencias.
%
% Genio y educación académica. Valor cultural.
%
% Etnia y registro vocal. Recomendación de transporte.
%
% TONADA SALTEÑA EN LA MELODÍA.
