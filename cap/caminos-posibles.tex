\chapter{Caminos posibles}
\label{cap:caminos}

Hay una buena noticia: una pieza del «Cuchi» Leguizamón ha sido rescatada. Para que ella pueda ser considerada el Himno de la Universidad Nacional de Salta hay que recorrer alguno de posibles caminos para allanar las dificultades que la pieza, por algunas características que posee, impone.

\section{Tonalidad y registro}
\label{sec:tonalidad-registro}

Entre esas características está, en primer lugar, la tesitura de la voz humana para la que fue escrita, que no condice con el registro de la media de la población de Salta (ni la mundial). Adaptar la pieza transportándola un tono y medio abajo, a la tonalidad de Re mayor ---recomendable para orquesta---, o dos tonos abajo, a la tonalidad de Re\bemoltxt\ mayor ---recomendable para banda de vientos--- es imprescindible para que la línea melódica no quede accesible solamente a voces naturalmente dotadas o educadas musical y técnicamente. Sin embargo, el transporte no es gratis: lo que en Fa mayor es tocable en el piano, tanto en Re como en Re\bemoltxt\ mayor cobra dificultades, especialmente en los intervalos de décima, ya que no es lo mismo tocar el intervalo \musncp[voffset=-1pt]{\key f \major \clef bass <f, a>} que tocar el intervalo \hbox{\musncp[voffset=-2pt]{\key d \major \clef bass <d, fis>},} el cual requiere de una mano más grande. Lo mismo sucede con el transporte a Re\bemoltxt\ mayor. De todos modos, estos transportes, lo recordemos, están siendo sugeridos para versiones orquestadas o arregladas para banda militar u orquesta sinfónica. Claro está que no deja de ser un dilema tener, o no, versiones de un himno en diferentes tonalidades. No recomendamos acá la diversidad. Debe elegirse una, y adaptar las cuestiones técnicas a las diversas y posibles instrumentaciones. Esto favorecerá la el aprendizaje y la memoria colectiva de la melodía.

\subsection{Versión transportada a Re mayor}
\label{subsec:transporte-re}

\lilypondfile[staffsize=11]{part/leguizamon-himno_unsa-re.ly}

\subsection[Versión transportada a Re\bemoltxt\ mayor]{Versión transportada a Re\bemol\ mayor}
\label{subsec:transporte-reb}

\lilypondfile[staffsize=11]{part/leguizamon-himno_unsa-reb.ly}


\subsection{Tonalidad recomendada}
\label{subsec:tonalidad-recomendada}

Como Re\bemoltxt\ tiene sus bemoles, \emph{Re mayor}, aunque no sea la más facilitadora para bandas de vientos ---como lo son las bandas militares---, es mi recomendación, ya que sí facilita el canto para una mayoría de personas perteneciente a nuestra sociedad, la salteña, donde por razones de conformación étnica encontramos un amplio segmento de voces femeninas clasificadas en la voz de \emph{mezzasoprano}\footnote{Una voz ni muy aguda ni muy grave.}, y lo propio en las voces masculinas inscriptas en el registro de \emph{barítono}\footnote{Al igual que las \emph{mezzosopranos} en las voces femeninas, los \emph{barítonos} poseen facilidad para sonidos intermedios, ni muy agudos como los de un \emph{tenor}, ni muy graves como los de un \emph{bajo}.}. A la vez, también Re mayor facilita la ejecución en instrumentos de cuerda frotada\footnote{Violines, violas, violoncellos, contrabajos.}, instrumentos que cuando se entiende a la orquesta sinfónica desde una óptica clásica conforman la base de dicho organismo musical, y a la vez son capaces de organizarse en agrupaciones de cámara como tríos (violín, viola, violoncello), cuartetos ( dos violines, viola, cello) o quinteto (cuarteto más contrabajo), agrupaciones éstas que pueden contratarse con mucho menos presupuesto y con más ofertas que una orquesta si se desea tener un evento en vivo donde presentar el \emph{Himno}, himno que por su carácter íntimo \emph{pide} ser interpretado con instrumentos que sin esfuerzo expresen ese carácter, como los son los instrumentos de cuerdas ---y no tanto los instrumentos de bronce, aunque con buenos intérpretes siempre todo es posible.

La versión para piano y voz, si se elige Re mayor como única tonalidad para todas las versiones del \emph{Himno} ---criterio éste que ejercería una labor docente en la enseñanza-aprendizaje de la melodía en la población sin instrucción musical---, debe ser adaptada en parte para subsanar los ya mencionados problemas de tocabilidad surgidos del transporte, siendo necesaria la adaptación de algunos intervalos para que la mano humana acceda a ejecutar sobre un teclado esta pieza. Por otro lado, no parece negociable la existencia de una versión para voz y pianos, siendo esta formación la original en la composición de Leguizamón y la que mejor expresa el carácter íntimo de la canción.


\section{Orquestación, o arreglo}
\label{sec:orquestacion-arreglo}

\emph{Orquestar} una pieza musical consiste en respetar tanto las notas como la textura de un original y presentarlo en una versión que viste a esa pieza original con los timbres de los instrumentos de una orquesta. Así, por ejemplo, si se orquesta el \emph{Himno a la Universidad Nacional de Salta}, composición original para canto y piano, la orquesta \emph{debe} re-presentar al piano de forma totalmente fiel en cuanto a notas se refiere, repartiendo los sonidos simultáneos entre los instrumentos de la orquesta que el orquestador decida utilizar en cada momento de la pieza.

\emph{Arreglar para orquesta} una música determinada implica más que una orquestación, ya que lo que se busca es una \emph{versión} de la composición original con pinceladas de composición orientadas a nutrir a la composición original de elementos compatibles con el lenguaje propio de la música sinfónica (si se trata de una orquesta sinfónica), o de cámara (si se trata de una agrupación de pocos instrumentos). Un arreglo, dependiendo de las decisiones artísticas del arreglador, puede orientarse hacia una versión completamente alejada del espíritu de la música original en un extremo, o hacia el máximo respeto por sostener dicho espíritu en la versión en el otro extremo. En el medio, por supuesto, de todo hay.

Al ser el \emph{Himno} una pieza de muy corta duración, aunque se decida ---y es necesario que así sea--- un \emph{ritornello} que repita toda la pieza para permitir el doble de letra de lo que el poema de Pérez posee, se necesita tanto la composición de un par de compases que encause el retorno al inicio (primera y segunda casilla en la escritura de la partitura) como, quizás, una ampliación de la introducción que será, al repetirse, también \emph{intermezzo}. Esto, evidentemente, no es orquestar, por lo que esta es nuestra recomendación: \emph{arreglar el \emph{Himno}, tanto para piano y voz como para orquesta sinfónica y conjuntos de cámara como cuarteto de cuerdas, por ejemplo.}

Condición \emph{sine qua non} para acometer cualquier arreglo es, antes que nada, tener definida una letra que dé marco semántico al arreglador para que éste pueda representarlo musicalmente en su trabajo.
